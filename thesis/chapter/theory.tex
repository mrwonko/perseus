\chapter{The basics of designing, compiling and executing programming languages}

% theoretische methodischen Grundlagen, die im weiteren Verlauf bei der Problemlösung im praktischen Teil eine wichtige Rolle spielen
% Detailgrad so hoch, wie für Verständnis nötig (Leser ist anonymer Informatiker)
% >= 25%

    \section{Designing programming languages}
    
        simplicity and power
        
        static vs dynamic typing
        
        paradigms
        
        type inference
        
        constness
        
        purity
    
    \section{Compilers}
    
        Chomsky hierarchy
        
        Lexer
        
        Parser -- LL, LR, GLR, ...
        
        Code generation: tail recursion? separate optimization subsection?
    
    \section{Virtual Machines}
    
        VM vs Native code
        
        Existing machines?
        
        \subsection{Details} % TODO: name
            fetch \& execute
            
            stack based vs registers
            
            opcodes \& choosing them
            
            coroutines % allgemein einführen
            
            error handling
            
            stack unwinding?
    
    \section{Patterns and features used}
    
        RAII
        
        Sum Types
        
        Optional
        
        Iterators
        
        Visitor
        
        Templates
        
        (Macros)
