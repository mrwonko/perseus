%%%%%%%%%%%%%%%%%%%%%%%%%%%%%%%%%%%%%%%%%%%%%%%%%%%%%%%
% Einführung: Wichtigster Teil (zusammen mit Schluss) %
%%%%%%%%%%%%%%%%%%%%%%%%%%%%%%%%%%%%%%%%%%%%%%%%%%%%%%%

% Einleitung
\chapter{Introduction}

%TODO Marcus: Abstract?

%TODO explain the language's name somewhere in this chapter?
    
    \section{Target audience}
    This thesis is written under the assumption that the reader has a solid understanding of the basics of computer science.
    % TODO more detail?
    
	\section{The raison d'être of programming languages}
	
	Software development is generally about getting good results quickly. Programming languages are a means to this end: They are supposed to make development easier by reducing the amount of work and preventing errors. This is an ongoing process and new programming languages are still developed to better fulfill certain tasks. This thesis is about implementing one such language the author envisioned: Perseus.

	\section{Goal} %TODO better title?
	% Ziel
	% konkrete Zielsetzung; auch: was nicht?
	% TODO
	
	Perseus is to be a strongly typed scripting language with native serialization support for execution contexts, so a paused script can be saved and later resumed, possibly on a different computer.
	
	This Thesis aims to create prototypes of a compiler and a virtual machine implementing a small subset of Perseus, mostly consisting of basic arithmetic and relational operations, functions and variables. Those prototypes can later be built upon to create a full-blown language, but that is outside of the scope of this thesis.
	
	The aim was for the following program to work -- as we shall see, this was achieved with the exception of one detail.
	
	\begin{perseuslisting}[caption={Desired target language example},label={lst:target_language}]
function fib( x : i32 ) -> i32
    // if the body is a single expression, no {}-block is required
    if x <= 1
        x
    else
        fib( x - 1 ) + fib( x - 2 )

impure function main()
{
	let index : i32 = 10;
	// variable types can be deduced from their initialization
	let result = fib( index );
	// print is an impure built-in function (i.e. it has side effects, in this case writing something), so main() needs to be impure as well to be able to call it
	print( result )
}
	\end{perseuslisting}
	
	The language itself is not of much interest in this thesis -- the focus is on the compiler and the virtual machine, not least because the language is still a work in progress.
    
    
	\section{Approach} %TODO better title
	% Beschreibung der Vorgehensweise -- wie soll Ziel erreicht werden? Welche Themen in welchen Kapiteln, warum? Prinzipielle Argumentationslinie? -> roten Faden deutlich machen
	
	After this introduction this thesis starts by recapitulating the necessary theory behind compilers and virtual machines and explaining the important software patterns used. It also looks into the available technology before it goes on to describe in detail the requirements -- what exactly are the compiler and the virtual machine supposed to do? Then it explains the design -- how those requirements were to be fulfilled -- followed by details on the implementation of said design. Finally the results are evaluated.
	
	%TODO sufficient?
	
	\section{Why a new programming language?}
	% Motivation
	% Einordnung der Thesis in allgemeinen Problemkontext -- diesen dazu kurz beschreiben (ca. 1--2 Seiten)
	
	There are plenty of programming languages to choose from -- why create yet another one? In short because there appears to be a niche that no language fills yet: A safe powerful strongly typed scripting language with elements from functional and object-oriented programming languages that has coroutines which can be serialized to a cross-platform binary format to be later resumed, possibly on a different computer. It should be easy to embed in {\CC} applications, have no garbage collection and avoid null- and dangling pointers.
	
	AngelScript\cite{angelscript} is an example of a statically typed scripting languages for embedding in {\CC}, but it doesn't have coroutines that can be serialized, has null-pointers and it's type system is not powerful enough, lacking generic types. And while Lua\cite{lua}, which embeds nicely in C, has coroutines that can be serialized using Pluto\cite{pluto}, it is dynamically typed and uses garbage collection. Rust\cite{rust} is similar in a lot of regards, though it too lacks in serialization, but it's no scripting language.
	
	Thus Perseus aims to fill this niche.
	
	\section{Why a new virtual machine?}
	
	To compile a new language a new compiler is required. But why use custom byte code on a custom virtual machine? Because there are no suitable existing virtual machines without garbage collection that support coroutines and stack inspection, which are required by the language, as is further explained in the \nameref{requirements} chapter.
	
	\section{Structure of this thesis}
	
	Chapter 2 is about the theory required to understand this thesis: Programming language concepts, compiler construction, virtual machines and the various software patterns used in the implementation. Chapter 3 highlights relevant existing compilers, virtual machines and programming languages. Chapter 4 lays down the requirements that have to be fulfilled by the compiler and the virtual machine this thesis is about. Chapter 5 explains the intended design for the compiler and the virtual machine: What language and libraries would be used? Which components would be required and how would they interact? Chapter 6 highlights interesting developments during implementation: Did anything go wrong? What are some interesting code excerpts? Chapter 7 assesses the results in relation to the requirements and chapter 8 proposes possible future improvements.
