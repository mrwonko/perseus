% Einleitung
\chapter{Introduction} % TODO name?

%*  Abstract (Summary)
%*  The need for a new programming language (Introduction)
%   *   Target audience
%   *   [Context]
%   *   [Goals]
%   *   [Approach/Summary: what did I do, generally?]
%
%Einleitung:
%In diesem Teil geht es um drei Aspekte. Zum einen soll ganz am Anfang die Themenstellung in einen allgemeinen Problemkontext eingeordnet werden, indem dieser kurz (ca. 1--2 Seiten) beschrieben wird. Davon ausgehend wird die konkrete Zielsetzung der Bachelorarbeit dargestellt. Was soll mit der Bachelorarbeit konkret erreicht werden? Welche Erkenntnis, welches Produkt, welche Software-L\"osung soll am Ende der Arbeit vorhanden sein? In diesem Teil kann auch erl\"autert werden, welche Aspekte bewusst aus der Arbeit ausgeschlossen werden. Als drittes Element der Einleitung kommt eine Beschreibung der Vorgehensweise hinzu. Sie beantwortet die Frage, wie das Ziel erreicht werden soll. Welche Themen werden in welchem Kapitel Leitlinien zum Betriebspraktikum und zu Bachelorarbeiten behandelt und warum? Wie ist die prinzipielle Argumentationslinie? Die Darstellung der Vorgehensweise bietet die Chance, den roten Faden durch die Arbeit deutlich zu machen und den Leser bereits auf diesen roten Faden einzustimmen.

    \section{\"Uberblick}
        \subsection{Spam}
        \subsection{Eggs}

    \section{Motivation}

    \section{Ziel}

    \section{\"Uberblick \"uber die Arbeit}