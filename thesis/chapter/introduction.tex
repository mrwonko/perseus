%%%%%%%%%%%%%%%%%%%%%%%%%%%%%%%%%%%%%%%%%%%%%%%%%%%%%%%
% Einführung: Wichtigster Teil (zusammen mit Schluss) %
%%%%%%%%%%%%%%%%%%%%%%%%%%%%%%%%%%%%%%%%%%%%%%%%%%%%%%%

% Einleitung
\chapter{Introduction} %TODO name?

%TODO explain the language's name somewhere in this chapter
    
    \section{Target audience}
    This thesis is written under the assumption that the reader has a solid understanding of the basics of computer science.
    % TODO more detail?

	\section{Goal} %TODO better title?
	% Ziel
	% konkrete Zielsetzung; auch: was nicht?
	% TODO
	
	I have a concept for a programming language called Perseus. I envision a strongly typed scripting language with native serialization support for execution contexts, so a paused script can be saved and later resumed, possibly on a different computer.
	
	My goal for this thesis was to create prototypes of a compiler and a virtual machine implementing a small subset of it, mostly consisting of basic arithmetic and relational operations, functions and variables. Those prototypes can later be built upon to create a full-blown language, but that is outside of the scope of this thesis.
	
	The aim was for the following program to work -- as we shall see, I achieved that with the exception of one quirk.
	
	\lstperseus
	\begin{lstlisting}[caption={Desired target language}]
function fib( x : i32 ) -> i32
    // if the function body is a single expression, no {}-delimited block is required
    if x <= 1
        x
    else
        fib( x - 1 ) + fib( x - 2 )

impure function main()
{
	let index : i32 = 10;
	// variable types can be deduced from their initialization
	let result = fib( index );
	// print is an impure built-in function (i.e. it has side effects, in this case writing something), so main() needs to be impure as well to be able to call it
	print( result )
}
	\end{lstlisting}
	
	I will not go into much detail on the language itself in this thesis -- the focus is on the compiler and the virtual machine, not least because the language is still a work in progress.
    
    
	\section{Approach} %TODO better title
	% Beschreibung der Vorgehensweise -- wie soll Ziel erreicht werden? Welche Themen in welchen Kapiteln, warum? Prinzipielle Argumentationslinie? -> roten Faden deutlich machen
	
	After this introduction I start by recapitulating the necessary theory behind compilers and virtual machines and explaining the important software patterns used in my solution. I also look into the available technology before I go on to describe in detail the requirements -- what exactly are the compiler and the virtual machine supposed to do? Then I explain my design -- how I planned to fulfill those requirements -- followed by details on the implementation of said design. Finally I evaluate my results, comparing them to my plans.
	
	%TODO sufficient?
	
	\section{Why a new programming language?}
	% Motivation
	% Einordnung der Thesis in allgemeinen Problemkontext -- diesen dazu kurz beschreiben (ca. 1--2 Seiten)
	%TODO
	
	\section{Why a new VM?}
	
	in short: serialization of coroutines
